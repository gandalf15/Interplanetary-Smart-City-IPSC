\chapter{Methodology \& Technologies\label{chap:technology}}
\quad The methodology, technology, development tools and project risk assessment are further described and addressed in this chapter.

\section{Methodology}
This project aims to explore the possibilities of utilizing blockchain technology, the best concepts from multiple peer-to-peer protocols and ideology of IPFS~\cite{labs_ipfs_nodate} for interplanetary smart cities. This topic currently gained a lot of interest and became active field of research. As we could read in the section Blockchain in IoT \& Smart Cities, there are many different approaches to address this problematic. Therefore, it is open ended research topic and the traditional software development techniques would not be suitable. Hence, I decided to use exploratory programming technique. These steps were followed in order to achieve progress and develop the application.

\begin{itemize}
\item Background literature reading and understanding the research topic.
\item Literature review of an existing similar research projects.
\item Developing a project plan
\item Learning about relevant software technologies, programming languages and tools used in this research area.
\item Creating functional and non functional requirements for the application.
\item Designing the system architecture that can solve the stated challenges.
\item Experimenting and exploring many potential technologies, languages and tools that could be suitable fort this application.
\item Implementing a prototype.
\item Developing the system and implementing functionalities needed to fulfill all requirements.
\item Evaluating and testing  the developed system.
\item Writing this report and user manual.
\item  Summarizing the project in conclusion and discussing future work.
\end{itemize}


\section{Technology}
\quad 
using golang because chaincode in hyperledger fabric is writen in it and for web server... Using javascript because it is an SDK for fabric, using solidity as a language for smart contracts on ethereum.

using IPFS as a storage solution (content addressed)

