\chapter{Requirements\label{chap:requirements}}
\quad The application's functional and nonfunctional requirements are specified in the following chapter. Both of them are sorted based on priority. The requirements with highest priority are listed first.

\section{Functional Requirements}
\textbf{FR} -> functional requirement  \newline
These are the functional requirements:

\begin{itemize}
\item \textbf{FR1 ->} The solution will be fully decentralised peer-to-peer network with different options of connectivity, including Internet, Wi-Fi, Ethernet, etc.
\item \textbf{FR2 ->} The nodes should be able to exchange data without a centralised authority in a safe, secure and confidential manner and in an accountable fashion.
\item \textbf{FR3 ->} The solution should record every transaction between nodes.
\item \textbf{FR4 ->} There will be provisions for two nodes to pair up adequately based on multiple parameters. For example freshness of the data, bandwidth, latency, etc.
\item \textbf{FR5 ->} The node should be able to publish what data it can provide.
\end{itemize}
\vspace{\baselineskip}

OLD LIST:
\begin{enumerate}
\item The application will be utilising fully distributed peer-to-peer network architecture.
\item The application will be able to function without connection to the Internet on a local network.
\item The nodes will be able to confidentially exchange (paid or not) data.
\item If the created P2P network is private then the owner of the network will have total control. ( over it despite it utilises peer-to-peer architecture.)
\item Upon the first launch of the application, it will try to find local nodes to bootstrap.
\item The application will try to connect to predefined trusted nodes (in case of available connection to the Internet) to bootstrap.
\item The node should prefer to download the same piece of data from physically closer nodes (be aware of physical distance).
\item The application should keep track of every transaction of data between nodes.
\item The user should be able to see node ID.
\item The user should be able to see (list) connected nodes.
\item The user should be able to see what data can nodes provide.
\item The user should be able to specify if a node can publish information what data can provide.
\item The user should access major functionality through web user interface.
\item The node should be able to prove that requested data were exchanged. (but still remain confidential)
\end{enumerate}

\vspace{\baselineskip}

\section{Non-Functional Requirements}
\textbf{NFR} -> nonfunctional requirement  \newline
These are the nonfunctional requirements:

\begin{itemize}
\item \textbf{NFR1 ->} The solution will be scalable.
\item \textbf{NFR2 ->} The solution will be robust. To be more specific, it will not fail as a result of individual components failing.
\item \textbf{NFR3 ->} The solution will be resistant to a number of attacks. For example Sybil, Eclipse, Churn and DDoS attack.
\end{itemize}
\vspace{\baselineskip}

OLD LIST:
\begin{enumerate}
\item The application will be scalable from single node up to \(10^8\) nodes.
\item The application will be robust.
\item The application will be reliable.
\item The application will be Byzantine fault tolerant.
\item The application will be will be resistant to Sybil attack.
\item The application will be will be resistant to Eclipse attack.
\item The application will be resistant to Churn attack.
\item The application will be resistant to Distributed Denial of Service (DDoS) attack.
\item The application will be resistant to Attacks on data storage.
\item The node should be able to exchange data fast.
\end{enumerate}

\vspace{\baselineskip}